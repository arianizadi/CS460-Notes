\documentclass[twocolumn]{article}
\usepackage[utf8]{inputenc}
\usepackage[skip=6pt, indent=0pt]{parskip}
\usepackage{graphicx}

\title{CS 460 {-} Compilers}
\author{Arian Izadi}
\date{Spring 2024}

\begin{document}

\maketitle

\section{Jan 22}

\subsection{Languages}

Syntax is the rules for what a syntacically correct program looks like.
Semantics is the meaining of a program.

When does it matter the order of evaluation (right to left vs left to right)?
When the code has side effects, an example of this is postfix vs prefix increment (a++ vs ++a).

Compilers for a language L, move from front end $\to$ intermediate representation $\to$ back end.

\begin{itemize}
  \item Front end: FIND IN BOOK
  \item Intermediate: FIND IN BOOK
  \item Back end: FIND IN BOOK
\end{itemize}

\subsection{Lexical Analysis \& Scanning}

Lexical analysis, a scanner, is the process of converting a stream of characters into a stream of tokens.

\begin{enumerate}
  \item Find all terminals in the grammar.
  \item Write the Scanner.
        \begin{enumerate}
          \item Do we use a DFA, NFA, or PDA{?}
          \item Look at token types. All tokens can be expressed by a regular expression.
                \begin{enumerate}
                  \item Symbols: Semicolon, commas, etc.
                  \item Keywords: for, while, etc.
                  \item Variables: x, y, etc.
                  \item Numbers: 1, 3.14, 0{x}64, etc.
                \end{enumerate}
        \end{enumerate}
\end{enumerate}

Chomsky Language Hierarchy

\begin{itemize}
  \item Type 0: Unrestricted (Turing Machines)
  \item Type 1: Context Sensitive
  \item Type 2: Context Free (PDA)
  \item Type 3: Regular Expressions (NFA, DFA)
\end{itemize}

Both RE and CFG have 1 non-terminal on the left of any combination of terminals and non-terminals on the right.

\textbf{Example 1:}

$S \to X \quad$
$X \to aXb | d \quad$
\hfill
not regular: $a^n d b^n$

\textbf{Example 2:}

$S \to X \quad$
$X \to aX | b \quad$
\hfill
regular: $a^* b$



\end{document}